\documentclass[10pt]{beamer}

% \usepackage{define}
\usepackage{animate}

\usetheme{CCFD}
\usepackage{color}
\definecolor{gray97}{gray}{.90}
\definecolor{gray75}{gray}{.75}
\usepackage{listings}
\lstset{frame=Ltb,
     framerule=0pt,
     aboveskip=0cm,
     framextopmargin=0pt,
     framexbottommargin=0pt,
     framexleftmargin=0cm,
     framesep=0pt,
     rulesep=0pt,
     backgroundcolor=\color{gray97},
     rulesepcolor=\color{black},
     language=C,
           basicstyle=\ttfamily\scriptsize,
           keywordstyle=\color{blue}\ttfamily,
           stringstyle=\color{red}\ttfamily,
           commentstyle=\color{green}\ttfamily,
          breaklines=true,
          }
\lstdefinestyle{consol}
   {basicstyle=\scriptsize\bf\ttfamily,
    backgroundcolor=\color{gray75},
}
\resetcounteronoverlays{lstnumber}

\newcommand{\tabitem}{%
  \usebeamertemplate{itemize item}\hspace*{\labelsep}}

\usepackage{tikz}
\usetikzlibrary{calc,shapes,arrows.meta}

\eventtitle{Computer Science I}
\title{Lecture 11\\Strings}
\date{}

\setbeamertemplate{blocks}[rounded][shadow=true]
\setbeamertemplate{navigation symbols}[]

\begin{document}

\frame{
    \titlepage
}

\section{Tests are coming}

\begin{frame}[fragile]
  \frametitle{Tests are coming}
  \framesubtitle{16.01 and 23.01}  
Know:
\begin{itemize}
  \item Everything from the first test
  \item Pointers and addresses
  \item Passing data to and from functions with pointers
  \item Static and dynamically allocated 1 and 2D arrays
  \item Arrays and functions
  \item File I/O operations
  \item Operations on strings
  \item Structures (next lecture)
\end{itemize}
Have:
  \begin{itemize}
    \item A Pen
    \item Student ID
  \end{itemize}
  \textbf{Do not} have:
  \begin{itemize}
    \item Notes
    \item Any electronic devices
  \end{itemize}

\end{frame}

\section{Operation on strings}

\begin{frame}[fragile]
  \frametitle{char, char* and char[]}
  \framesubtitle{16.01 and 23.01}  
\centering
  \begin{itemize}
    \item Write a program illustrating all the characters - some we shell not see
    \item Find the special character, we will need it later
    \item Write a program that uses an array of characters to store a word or a sentence
    \item Store something in an array, try \%s to print the content
    \item Use the special character, it is a termination sign
  \end{itemize}

\end{frame}

\begin{frame}[fragile]
  \frametitle{String is an array of char}
  \framesubtitle{with a special, terminating character added at the end}  
\centering
  \begin{itemize}
    \item Innitialize a string in a more convenient manner
    \item How much space does a string occupies?
    \item What is the last character?
    \item Do I need to provide the size if an array is static?
    \item What is a difference in using char[] and char*?
    \item Why can't we reassign some strings?
  \end{itemize}

\end{frame}

\begin{frame}[fragile]
  \frametitle{Reading strings from keybord}
  \framesubtitle{scanf?}  
\centering
  \begin{itemize}
    \item Write a program reading a string from a keyboard
    \item Can we do better?
    \item char *fgets (char *str, int size, FILE* file);
    \item We can also read from file!
  \end{itemize}

\end{frame}

\begin{frame}[fragile]
  \frametitle{Manipulating null terminating strings}
  \framesubtitle{string.h}  
\centering
  \begin{itemize}
    \item Comparison: int strcmp ( const char *s1, const char *s2 );
    Returns 0 if s1 and s2 are the same; less than 0 if s1<s2; greater than 0 if s1>s2.
    \item String concatenate: char *strcat ( char *dest, const char *src );
    \item Copy: char *strcpy ( char *dest, const char *src );
    \item Length of a string: int strlen ( const char *s );
     \item char* strchr(s1, ch);
     Returns a pointer to the first occurrence of character ch in string s1.
   	\item char* strstr(s1, s2); Returns a pointer to the first occurrence of string s2 in string s1.
  \end{itemize}

\end{frame}

\end{document}
