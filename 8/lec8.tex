\documentclass[10pt]{beamer}

% \usepackage{define}
\usepackage{animate}

\usetheme{CCFD}
\usepackage{color}
\definecolor{gray97}{gray}{.90}
\definecolor{gray75}{gray}{.75}
\usepackage{listings}
\lstset{frame=Ltb,
     framerule=0pt,
     aboveskip=0cm,
     framextopmargin=0pt,
     framexbottommargin=0pt,
     framexleftmargin=0cm,
     framesep=0pt,
     rulesep=0pt,
     backgroundcolor=\color{gray97},
     rulesepcolor=\color{black},
     language=C,
           basicstyle=\ttfamily\scriptsize,
           keywordstyle=\color{blue}\ttfamily,
           stringstyle=\color{red}\ttfamily,
           commentstyle=\color{green}\ttfamily,
          breaklines=true,
          }
\lstdefinestyle{consol}
   {basicstyle=\scriptsize\bf\ttfamily,
    backgroundcolor=\color{gray75},
}
\resetcounteronoverlays{lstnumber}

\newcommand{\tabitem}{%
  \usebeamertemplate{itemize item}\hspace*{\labelsep}}

\usepackage{tikz}
\usetikzlibrary{calc,shapes,arrows.meta}

\eventtitle{Computer Science I}
\title{Lecture 8\\1D Arrays\\{\tiny and other disasters}}
\date{}

\setbeamertemplate{blocks}[rounded][shadow=true]
\setbeamertemplate{navigation symbols}[]

\begin{document}

\frame{
    \titlepage
}

\section{Today}
\begin{frame}
  \frametitle{Today}
  \framesubtitle{}
  
  Today:
  \begin{itemize}
		\item File I/O - continue
		\item A program that preperas points for plotting a function
		\item A program that reads data from a file, and manipulates them.
		\item A program that generates random numbers and stores them in a file.
		\item A program that reads a file, and calculates an average
		\item A program that reads a file, sorts it and stores back
  \end{itemize}
\end{frame}

\section{Files}

\begin{frame}[fragile]
  \frametitle{Files}
FILE structure to handle files:
\begin{lstlisting}
FILE *fp;
\end{lstlisting}

To open a file use \textit{fopen()}:
\begin{lstlisting}
FILE *fopen(const char *filename, const char *mode);
//e.g.:
fp=fopen("c:\\test.txt", "r");
\end{lstlisting}

To close a file use \textit{fclose()}:
\begin{lstlisting}
int fclose(FILE *a_file);
//e.g.:
fclose(fp);
\end{lstlisting}
 
\end{frame}

\begin{frame}[fragile]
  \frametitle{Files}
  \framesubtitle{fopen modes}
  Depending on what we require the file to:
\begin{itemize}
	\item r  - open for reading
	\item w  - open for writing (file need not exist)
	\item a  - open for appending (file need not exist)
	\item r+ - open for reading and writing, start at beginning
	\item w+ - open for reading and writing (overwrite file)
	\item a+ - open for reading and writing (append if file exists)
\end{itemize}



\end{frame}

\begin{frame}[fragile]
  \frametitle{Files}
  \framesubtitle{Reading and writing with fprintf, fscanf}
Printing to file:
\begin{lstlisting}
FILE *fp;
fp=fopen("c:\\test.txt", "w");
fprintf(fp, "Testing...\n");

...
fclose(fp);
\end{lstlisting}

Reading from file:
\begin{lstlisting}
FILE *fp;
fp=fopen("c:\\test.txt", "w");
int a;
fcanf(fp, "%d", &a);
...
fclose(fp);
\end{lstlisting}
 
\end{frame}

\subsection{Examples}

\begin{frame}
  Use static arrays only.
  \frametitle{Examples}
  \begin{enumerate}
  \item Write a program that writes to a file coordinates to plot $f(x)=sin(x)$
for a range $<0,2\pi>$
  \item Write program that reads points coordinates from a file and decides if those are in a circle of radius 1.
  \item Write a program that generates N random numbers and stores them to a file.
  \item Write a program that reads a data file, calculates an average value and finds the number of elements above, and below that average.
  \item Write a program that reads values from a file, sorts them and stores them to a new file.
  \item Example test questions
\end{enumerate}
  
\end{frame}

\end{document}
