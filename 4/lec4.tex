\documentclass[10pt]{beamer}

% \usepackage{define}

\usetheme{CCFD}
\usepackage{color}
\definecolor{gray97}{gray}{.90}
\definecolor{gray75}{gray}{.75}
\usepackage{listings}
\lstset{frame=Ltb,
     framerule=0pt,
     aboveskip=0cm,
     framextopmargin=0pt,
     framexbottommargin=0pt,
     framexleftmargin=0cm,
     framesep=0pt,
     rulesep=0pt,
     backgroundcolor=\color{gray97},
     rulesepcolor=\color{black},
     language=C,
           basicstyle=\ttfamily\scriptsize,
           keywordstyle=\color{blue}\ttfamily,
           stringstyle=\color{red}\ttfamily,
           commentstyle=\color{green}\ttfamily,
          breaklines=true,
          }
\lstdefinestyle{consol}
   {basicstyle=\scriptsize\bf\ttfamily,
    backgroundcolor=\color{gray75},
}
\resetcounteronoverlays{lstnumber}

\newcommand{\tabitem}{%
  \usebeamertemplate{itemize item}\hspace*{\labelsep}}

\usepackage{tikz}
\usetikzlibrary{calc,shapes}

\eventtitle{Computer Science I}
\title{Lecture 4}
\date{}

\setbeamertemplate{blocks}[rounded][shadow=true]
\setbeamertemplate{navigation symbols}[]

\begin{document}

\frame{
    \titlepage
}

\begin{frame}[fragile]
  \frametitle{printf() and scanf()}

\begin{lstlisting}
#include <stdio.h>
int a;
double b;
scanf("%d", &a);
scanf("%lf", &b);

printf("\t a=%d \t b=%lf \n", a, b);

\end{lstlisting}

\end{frame}

\section{Branching}
% if ?: switch

\begin{frame}
  \begin{itemize}
    \item Branching is deciding what actions to take
    \item The program chooses to follow one branch or another
    \item \textit{if()}
    \item \textit{? :} operator
    \item \textit{switch() {}}
  \end{itemize}
\end{frame}

\subsection{if()}

\begin{frame}
  \frametitle{\textit{if()}}
  \framesubtitle{If today is Monday I will study C programming}
  
  \begin{itemize}
    \item Based on a concept of \textit{TRUE} or \textit{FALSE}
    \item \textit{TRUE} is a statement that evaluates to a nonzero value
    \item \textit{FALSE} evaluates to zero
    \item Use of relational operators:
    \begin{itemize}
      \item $>$     greater than       -       5 $>$ 4 is TRUE
      \item $<$     less than           -      4 $<$ 5 is TRUE
      \item $>=$    greater than or equal -    4 $>=$ 4 is TRUE
      \item $<=$    less than or equal      -  3 $<=$ 4 is TRUE
      \item $==$    equal to               -   5 $==$ 5 is TRUE
      \item $!=$    not equal to            -  5 $!=$ 4 is TRUE
    \end{itemize}
    \item exapmles ...
  \end{itemize}
  \centering
  {\bf \color{red} \Large Do not use \textit{=} to test equality, use == !!!} 
\end{frame}

\begin{frame}
  \frametitle{AND and OR}
  \framesubtitle{\&\& and $||$}
  
  \begin{itemize}
    \item Used for more complex logical statements
    \item \&\& - logical AND
    \item $||$ - logical OR
  \end{itemize}
\end{frame}

\begin{frame}[fragile]
  \frametitle{Basic if syntax}
  
  \begin{lstlisting}
if (statement that evaluates to TRUE or FALSE)
  instruction
  
if (statement that evaluates to TRUE or FALSE)
{
  multi
  line
  instruction
}  
    \end{lstlisting}
    
    examples ...
\end{frame}

\begin{frame}[fragile]
  \frametitle{What else?}
  
  \begin{lstlisting}
if (statment that evaluates to TRUE or FALSE)
  instruction
else
  another set of instructions
  
if (statment that evaluates to TRUE or FALSE)
{
  multi
  line
  instruction
}
else
  another set of instructions - could be multiline
    \end{lstlisting}
    
    examples ...
\end{frame}

\begin{frame}[fragile]
  \frametitle{else if()}
  
  \begin{lstlisting}
if (statment that evaluates to TRUE or FALSE)
  instruction
else if()
  another set of instructions
else if()
 ...
  
if (statment that evaluates to TRUE or FALSE)
{
  multi
  line
  instruction
}
else if()
  another set of instructions - could be multiline
    \end{lstlisting}
    
    examples ...
\end{frame}

\subsection{Inline if}

\begin{frame}[fragile]
  \frametitle{Inline if}
  \framesubtitle{?:}
  \begin{itemize}
    \item It is like an \textit{if else}
    \item Might be used within expressions
    \item The only ternary operator in C
  \end{itemize}
  \begin{lstlisting}
if condition is true ? then X return value : otherwise Y value;

int a=5;
int b=1, c=2;
int d = b > c ? a + b : a + c;
    \end{lstlisting}
    
    examples ...
\end{frame}

\subsection{swich}

\begin{frame}[fragile]
  \frametitle{switch()}
  \begin{itemize}
    \item Much like nested if else
    \item Might be more efficient
  \end{itemize}
  
    \begin{lstlisting}
switch( expression )
{
  case expr1:
    instructions;
    break;
  case expr2:
    instructions;
    break;
  default:
    instructions;
}
    \end{lstlisting}
    
    \begin{itemize}
      \item key word \textit{break}
      \item key word \textit{default}
      \item examples...
    \end{itemize}
\end{frame}

















\end{document}
