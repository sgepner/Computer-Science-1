\documentclass[10pt]{beamer}

% \usepackage{define}

\usetheme{CCFD}
\usepackage{color}
\definecolor{gray97}{gray}{.90}
\definecolor{gray75}{gray}{.75}
\usepackage{listings}
\lstset{frame=Ltb,
     framerule=0pt,
     aboveskip=0cm,
     framextopmargin=0pt,
     framexbottommargin=0pt,
     framexleftmargin=0cm,
     framesep=0pt,
     rulesep=0pt,
     backgroundcolor=\color{gray97},
     rulesepcolor=\color{black},
     language=C,
           basicstyle=\ttfamily\scriptsize,
           keywordstyle=\color{blue}\ttfamily,
           stringstyle=\color{red}\ttfamily,
           commentstyle=\color{green}\ttfamily,
          breaklines=true,
          }
\lstdefinestyle{consol}
   {basicstyle=\scriptsize\bf\ttfamily,
    backgroundcolor=\color{gray75},
}
\resetcounteronoverlays{lstnumber}

\newcommand{\tabitem}{%
  \usebeamertemplate{itemize item}\hspace*{\labelsep}}

\usepackage{tikz}
\usetikzlibrary{calc,shapes}

\eventtitle{Computer Science I}
\title{Lecture 3}
\date{}

\setbeamertemplate{blocks}[rounded][shadow=true]
\setbeamertemplate{navigation symbols}[]

\begin{document}

\frame{
    \titlepage
}

\frame{
  \frametitle{We know types!}
    \begin{itemize}
      \item int,
      \item float, double
      \item char,
      \item bool
      \item void
      \item example ...
    \end{itemize}
}

\section{Working with variables}

\begin{frame}[fragile]
  \frametitle{Variables}
  \centering
Substitution and assignment:
    \begin{lstlisting}
int a,b; //declare two variables of type int
a=35; //a store value of 35
b=6; //b store 6
a=a+b;// perform addition in temporary space, copy on to a
    \end{lstlisting}

An arithmetic expresion:
    \begin{lstlisting}
double x1; //declare a variable of type double
x1=(-b+sqrt(delta))/(2.0*a); //Perform RHS operation, write result to x1
    \end{lstlisting}

Example ...

\end{frame}

\subsection{Operators}

\begin{frame}[fragile]
  \frametitle{Substitution}
  \centering
  When using "=" sign variable on the LHS is assigned value of RHS. This does not necessarily means equality!\\
  The RHS is calculated first and later the value is copied to the LHS. Types of LHS and RHS should be the same.\\
  Mixing of types should be avoided.\\
\begin{lstlisting}
double x1=6.28;
int a = 2
a = x1; //loss of data since a=6!
x1=a;
\end{lstlisting}
There is an explicit way to change the type: casting
\begin{lstlisting}
double x1=6.28;
int a = 2
a = (int)x1; //loss of data, but no warning
x1=(double)2/3; //x1 is not zero
\end{lstlisting}


\end{frame}

\begin{frame}[fragile]
  \frametitle{Precedence of operators}
  \framesubtitle{The ones we know so far}
  \centering
  \begin{enumerate}
    \item () brackets
    \item + - unary plus/minus: (-1)
    \item * / \% binary operator a*b
    \item - + binary operator a+b
  \end{enumerate}

\begin{lstlisting}
-5 * 3 + 4 * 5. / 2. 
((-5)*3)+(4*5)/2.
\end{lstlisting}

examples ...

\end{frame}

\begin{frame}[fragile]
  \frametitle{Increment/decrement operators}
  Increment / decrement operators are unary operators that change the value of a variable by 1.\\
  They can have postfix or prefix form

\begin{lstlisting}
a++ //postfix
a--
++a //prefix
--a
\end{lstlisting}

\begin{lstlisting}
int a = 1;
int b = a++; // stores 1+a (which is 2) to a
             // returns the value of a (which is 1)
             // After this line, b == 1 and a == 2
a = 1;
int c = ++a; // stores 1+a (which is 2) to a
             // returns 1+a (which is 2)
             // after this line, c == 2 and a == 2
\end{lstlisting}

example ...

\end{frame}

\begin{frame}[fragile]
  \frametitle{Compound Assignment Operators += -= *= /=}
  Change the value of the RHS by the value of the LHS

\begin{lstlisting}
a+=b; // same as a = a +b;
a-=b; // same as a = a -b;
a*=b; // same as a = a *b;
a/=b; // same as a = a /b;
\end{lstlisting}

\begin{lstlisting}
int a = 1;
int b = 5;
a+=b;
a-=1;
a*=9;
a/=b;
...
\end{lstlisting}

example ...

\end{frame}

\section{IO operations}
\subsection{printf}

\begin{frame}[fragile]
  \frametitle{printf()}
  \begin{columns}
    \begin{column}{0.5\textwidth}
\begin{lstlisting}
int a = 1;
float b=4.78;
double c=3.4e9;
char d='e'
printf("%d %f %e %c \n", a, b, c, d)
\end{lstlisting}
    \end{column}
    \begin{column}{0.5\textwidth}
      \begin{itemize}
        \item \textit{\#include $<$stdio.h$>$}
        \item Sends formatted output to stdout - the screen
        \item \textit{int printf(const char *f, ...)}\\
        returns the total number of characters written
        \item \textit{f}-text to be written, might contain format tags, replaced by value provided in arguments. Format tag is as follows: \textit{\textbf{\%specifier}}
        \item We know some specifiers:\\
        c, d or i, e, E, f, g, o, u, x or X
        \item examples ...
      \end{itemize}
    \end{column}
  \end{columns}
\end{frame}

\begin{frame}[fragile]
  \frametitle{printf()}
  \frametitle{new stuff}
  \begin{columns}
    \begin{column}{0.5\textwidth}
\begin{lstlisting}
int a = 1;
float b=4.78;
double c=3.4e9;
char d='e'
printf("%+d %-f %10e %c \n", a, b, c, d)
printf("%.10d %10.10f %10e %c \n", a, b, c, d)
\end{lstlisting}
    \end{column}
    \begin{column}{0.52\textwidth}
      \begin{itemize}
        \item Format tag might contain more information
        \item \textit{\bf{\%[flags][width][.prec]spc}}
        \item Some Flags: - left justify, +force sign
        \item width - minimum number of characters to be printed
        \item .prec - for ints: the minimum number of digits to be written\\
        For e, E and f: the number of digits to be printed after the decimal point.\\
        For g and G: This is the maximum number of significant digits to be printed.
        \item examples ...
      \end{itemize}
    \end{column}
  \end{columns}
\end{frame}

\begin{frame}[fragile]
  \frametitle{printf()}
  \frametitle{The cursor control}
  \begin{columns}
    \begin{column}{0.5\textwidth}
\begin{lstlisting}
printf(" aaa b\b \n fff \t ff \v fff \r a ")
\end{lstlisting}
    \end{column}
    \begin{column}{0.52\textwidth}
      \begin{itemize}
        \item \textbackslash b - backspace
        \item \textbackslash n - carriage return and newline
        \item \textbackslash t - tab
        \item \textbackslash v - newline
        \item \textbackslash r - carriage return
        \item \textbackslash \textbackslash - \textbackslash
        \item \textbackslash '' - '
        \item examples ...
      \end{itemize}
    \end{column}
  \end{columns}
\end{frame}

\subsection{scanf}

\begin{frame}[fragile]
  \frametitle{scanf()}
  \begin{columns}
    \begin{column}{0.5\textwidth}
\begin{lstlisting}
int a;
scanf("%d", &a);//Read int
\end{lstlisting}
\centering
\includegraphics[width=0.7\textwidth, height=0.7\textheight, keepaspectratio]{scanf_perils}\\
He forgot abouth \& \\
{\tiny Source: the XKCD commic}
    \end{column}
    \begin{column}{0.5\textwidth}
      \begin{itemize}
        \item \textit{\#include $<$stdio.h$>$}
        \item Reads formatted input from stdin - the keybord
        \item \textit{int scanf(const char *f, ...)}\\
        returns the total number of characters written
        \item \textit{f}- a string containing a format specifier \textit{\textbf{\%specifier}}
        \item \textbf{Mind the \&} - very very important!!
        \item examples ...
      \end{itemize}
    \end{column}
  \end{columns}
\end{frame}

\section{math.h}

\begin{frame}[fragile]
  \frametitle{math.h}
  \framesubtitle{trigonometric}
  \begin{itemize}
  \item Defines various mathematical functions
  \item examples ...
  \end{itemize}

\begin{lstlisting}
#include <math.h>
double acos(double x) Returns the arc cosine of x in radians.
double asin(double x) Returns the arc sine of x in radians.
double atan(double x) Returns the arc tangent of x in radians.
double atan2(doubly y, double x)Returns the arc tangent in radians of y/x based on the signs of both values to determine the correct quadrant.

double cos(double x) Returns the cosine of a radian angle x.
double cosh(double x) Returns the hyperbolic cosine of x.
double sin(double x) Returns the sine of a radian angle x.
double sinh(double x) Returns the hyperbolic sine of x.
double tanh(double x) Returns the hyperbolic tangent of x.
\end{lstlisting}

\end{frame}

\begin{frame}[fragile]
  \frametitle{math.h}

\begin{lstlisting}
#include <math.h>
double exp(double x) Returns the value of e raised to the xth power.
double log(double x)Returns the natural logarithm (base-e logarithm) of x.
double log10(double x)Returns the common logarithm (base-10 logarithm) of x.
double pow(double x, double y)Returns x raised to the power of y.
double sqrt(double x)Returns the square root of x.
double ceil(double x)Returns the smallest integer value greater than or equal to x.
double fabs(double x)Returns the absolute value of x.
double floor(double x)Returns the largest integer value less than or equal to x.

\end{lstlisting}

for $e^x$ use \textit{exp(x)}, never use \textit{pow(exp(1), x)} ...

Examples!

\end{frame}

\end{document}
